\documentclass{article}


\usepackage[italian]{babel}
\usepackage{graphicx}
\usepackage{booktabs}

\usepackage[letterpaper,top=2cm,bottom=2cm,left=3cm,right=3cm,marginparwidth=1.75cm]{geometry}

% Useful packages
\usepackage{amsmath}
\usepackage{graphicx}
\usepackage[colorlinks=true, allcolors=blue]{hyperref}

\title{Appunti di Selvicoltura Generale}
\author{Luca Andreetta \\
\href{mailto:luca.andreetta.1@studenti.unipd.it}{luca.andreetta.1@studenti.unipd.it} }
\date{}
\begin{document}
\maketitle

%\begin{abstract}
%Your abstract.
%\end{abstract}
\section*{Premessa}
Questo documento contiene i miei appunti inerenti al corso di selvicoltura generale, tenuto dal Prof. Lingua.\\
Per mia semplicità e velocità di scrittura non sono presenti immagini.\\
Questi appunti sono da utilizzare come supporto alle slide del professore e al libro ``Selvicoltura Generale" di Piussi. 

\section{Generalità}
\subsection{Cos'è la selvicoltura?}

Selvicoltura = \textbf{selvi} (selva, bosco) + \textbf{coltura} (coltivazione)\\
\textit{``Scienza e pratica di coltivare il bosco, applicando i principi di ecologia forestale, rinnovazione naturale e razionali interventi per condizionare la struttura e la composizione specifica dei popolamenti forestali."}\\
Lo scopo della selvicoltura, oltre a creare reddito, non è quello di far crescere degli alberi, ma di creare un ecosistema.\\
Inoltre, la selvicoltura tende ad operare come farebbe naturalmente l'ambiente, velocizzando i processi naturali.\\
Con ``coltura" si intende il complesso di tecniche relative allo sfruttamento di una risorsa naturale, elaborate dall'uomo per conseguire una produzione che soddisfi le sue necessità.\\
Il termine bosco ha definizioni diverse, a seconda del contesto considerato:
\begin{itemize}
    \item Istat: $>0.5 \, \text{ha}$ di superficie e $>50\%$ di copertura arborea;
    \item Inventario Forestale: $>2000 \, \text{m}^2$ di superficie, $>20 \, \text{m}^2$ di copertura arborea e $>20 \, \text{m}$ di larghezza; inoltre, tale superficie viene considerata anche se sprovvista di copertura arborea (causa accidentale o utilizzazione);
    \item INFC: $>0.5 \, \text{ha}$ di superficie, $>10\%$ di copertura arborea e $>5 \, \text{m}$ di altezza minima a maturità in situ;
    \item testo unico TUFF: $>2000 \, \text{m}^2$ di estensione, $>20 \, \text{m}$ di larghezza e $>20\%$ di copertura arborea. La definizione veneta si allinea a questa nazionale.
\end{itemize}
\subsection{Bosco come ecosistema}
Durante la gestione forestale, occorre considerare non solo gli alberi, ma n° fattori contenuti nel sistema globale.\\
L'insieme degli alberi crea un microclima (il ``clima del bosco"), diverso dall'esterno: tale microclima permette la rinnovazione forestale.

\subsection{Le foreste in Italia}
\begin{itemize}
    \item Al 2015 occupavano il 36\% della superficie nazionale;
    \item la proprietà forestale è maggiormente privata, ovvero per piccole particelle. La maggior parte delle foreste sono a ceduo (cioè boschi ``poveri");
    \item la provvigione media è di $165 \, \frac{\text{m}^3}{\text{ha}}$ e l'area basimetrica media è di $22 \, \frac{\text{m}^2}{\text{ha}}$;
    \item l'incremento di legname annuo è pari a $3 \, \frac{\text{m}^3}{\text{ha}}$;
    \item la necromassa media è di $15 \, \frac{\text{m}^3}{\text{ha}}$ (il target sufficiente per mantenere livelli sufficienti di biodiversità è $20-40 \, \frac{\text{m}^3}{\text{ha}}$).
\end{itemize}

\subsection{Sistemi selvicolturali}
Insieme di azioni attuate per la coltivazione e rinnovazione del bosco. Si differenziano per i metodi di rinnovazione, popolamento prodotto ed organizzazione cronologica dei tagli.
\subsection{Funzioni del bosco}
\begin{itemize}
    \item Utilizzazioni: raccolta di un prodotto (materiale) da parte dell'uomo;
    \item Servizi: benefici che derivano all'uomo per la sola presenza del bosco;
    \item Usi: prevedono l'ingresso dell'uomo in bosco, non necessariamente per l'attività di raccolta o asportazione.
\end{itemize}

\section{Caratteristiche e sviluppo della pianta}
\subsection{Assortimenti}
Prodotti (tendenzialmente) legnosi che escono dal bosco. Ogni elemento con valore economico è un assortimento.\\
A seconda dell'assortimento d'interesse, cambia il metodo selvicolturale.\\
Dall'assortimento principale si ottengono assortimenti secondari (di minore qualità).
\begin{itemize}
    \item 1° toppo: è l'assortimento principale, dev'essere lungo, senza rami e con un grande diametro. Viene valorizzato con sfogliatura o con la tranciatura;
    \item 2° toppo: è il tronco con i nodi e viene valorizzato con la segagione;
    \item ramaglia: viene valorizzata mediante la triturazione o come biomassa per cellulosa/ardere.
\end{itemize}

\href{https://creafuturo.crea.gov.it/wp-content/uploads/2023/01/Figura1.jpg}{Utilizzazione degli assortimenti dell'albero}\\

Gli assortimenti forestali possono anche non essere legnosi (frutti di bosco, funghi,\dots).\\
Le utilizzazioni selvicolturali sono composte da 4 step: abbattimento, allestimento, concentramento ed esbosco.
\subsection{Portamento della pianta}
Si basa sul portamento del singolo albero e sulle relazioni tra diversi alberi.\\
Il portamento è la forma che assume la pianta; in condizioni reali è influenzato solo dal genotipo, mentre il fenotipo è influenzato dal clima, suolo e collocazione dell'albero.\\
Il \textbf{portamento della chioma} è dovuto dal 
\begin{itemize}
    \item tipo di foglie del dato albero (di luce o d'ombra);
    \item dominanza delle gemme apicali o laterali (monopodiale o simpodiale);
\end{itemize} 
Maggiore è la densità del popolamento e maggiore sarà l'altezza d'inserzione dei rami (con una maggiore autopotatura).\\
Il \textbf{portamento delle radici} è un parametro importante per la stabilità della pianta e per l'assunzione di acqua e nutrienti dal suolo.\\
Le radici possono essere fascicolate, fittonanti o superficiali.\\
Il \textbf{fusto} ha funzioni di trasporto, sostentamento e riserva. Ogni fusto può essere suddiviso in tre sezioni, a seconda della loro forma:
\begin{itemize}
    \item la parte basale ha la forma a neloide;
    \item la parte centrale è prettamente cilindrica;
    \item la sezione dove c'è l'inserzione dei rami è conoide.
\end{itemize}
L'obiettivo del forestale è massimizzare (mediante i trattamenti selvicolturali) la parte cilindrica del fusto, contemporaneamente considerando l'importanza che i rami hanno per la stabilità della pianta.\\

\subsection{Accrescimento}
La gestione forestale ruota intorno all'accrescimento diametrico ed ipsometrico.\\
Nell'estremità del fusto (o del ramo) in sviluppo sono presenti 3 zone: 
\begin{itemize}
    \item zona di accrescimento per divisione: le cellule del meristema si dividono e producono nuovi elementi; 
    \item zona di distensione e differenziazione: maggiore responsabile dell'allungamento del giovane fusto/ramo, le cellule vanno incontro ad un processo di distensione e differenziazione;
    \item zona di struttura primaria: le cellule sono morfologicamente e funzionalmente adulte. 
\end{itemize}
L'accrescimento radiale avviene per merito del meristema secondario, detto cambio cribo-vascolare; vengono prodotte cellule sia esternamente che internamente (tessuto vascolare e periderma). Il libro (floema) ed il legno sono il risultato dell'accrescimento secondario.\\
L'accrescimento dipende dalla disponibilità di risorse (fertilità) e dalla gerarchia di allocazione; è il risultato tra la fotosintesi lorda e la respirazione.\\
Negli alberi, l'allocazione delle risorse segue una precisa gerarchia: 
\begin{enumerate}
    \item nuove foglie e gemme;
    \item radici e radichette;
    \item riserve;
    \item cambio (accrescimento di tronco e rami); 
    \item sostanze di difesa ed allelopatiche;
\end{enumerate}
Il processo di riproduzione (frutti e semi) non viene considerato, poiché avviene in modo efficace solo ad intervalli temporali.
Questa gerarchia di allocazione può variare nel caso di:
\begin{itemize}
    \item suoli aridi o poco fertili: la priorità viene data all'apparato radicale;
    \item abbondanza di acqua ed azoto: vengono privilegiate le foglie;
    \item costanti sollecitazioni meccaniche: viene privilegiata la produzione di legno;
    \item ridotto trasporto floematico e stress idrico prolungato: viene stimolata la fioritura. 
\end{itemize}
L'accrescimento legnoso epigeo viene influenzato da:
\begin{itemize}
    \item fattori ambientali: clima stazionale (disponibilità di acqua e temperature favorevoli) e fertilità del suolo (disponibilità di nutrienti);
    \item competitività del singolo albero: è più difficile da regolare.
\end{itemize}
Generalmente, l'efficienza di crescita decresce con l'aumentare dell'età.
\subsubsection{Accrescimento longitudinale}
E' dovuto all'attività dei meristemi primari, presenti nelle gemme apicali.\\
Alle nostre latitudini segue l'andamento stagionale.\\
L'altezza complessiva dell'albero è limitata dall'efficienza del sistema di trasporto.\\
Nei cedui, la culminazione dell'incremento corrente in altezza avviene nei primissimi anni.\\
I diradamenti non hanno effetto incrementale sulla crescita in altezza, poiché non è un parametro sito-dipendente.
\subsubsection{Accrescimento diametrale}
E' dipendente dall'attività del cambio, che produce xilema e floema.\\
Alle nostre latitudini, il cambio ha attività stagionale: dormienza invernale ed attivazione in primavera. Nell'arco stagionale si mantiene in genere più a lungo dell'accrescimento longitudinale.\\
L'accrescimento radiale è molto sensibile alla densità del bosco ed alla posizione sociale.\\
\subsubsection{Accrescimento volumetrico}
A parità di altezza, l'aumento di diametro comporta un incremento di volume.\\
La formula per il calcolo del volume è:
\begin{equation}
    V = g \cdot h \cdot f
\end{equation}

\subsection{Sviluppo}
Insieme di trasformazioni qualitative che accompagnano la pianta attraverso i diversi stadi vitali; è intesa come una maturazione.\\
\begin{enumerate}
    \item di plantula: stato infantile, caratterizzato dalla presenza di cotiledoni;
    \item giovanile: sostenuto accrescimento longitudinale;
    \item di fertilità: espansione della chioma e raggiungimento della fase/capacità riproduttiva;
    \item di maturità: culminazione dell'incremento volumetrico;
    \item di vecchiaia: declino degli incrementi, calo della fruttificazione e diradamento della chioma;
    \item morte: di mortalità naturale.
\end{enumerate}
La morte può avvenire per diversi motivi:
\begin{itemize}
    \item eccesso di respirazione rispetto alla fotosintesi: diminuzione dell'efficienza fotosintetica per il diradarsi della chioma, mentre la percentuale di biomassa legnosa prodotta aumenta; la respirazione non può essere soddisfatta e la pianta muore per denutrizione;
    \item problemi di trasporto xilematico e floematico (cavitazione dei vasi ed embolie);
    \item stabilità meccanica che viene compromessa dalla dimensione;
    \item fattori esterni: come parassiti, siccità, fuoco,\dots
\end{itemize}
Paradossalmente, la longevità è inversamente proporzionale alla fertilità del suolo, poiché varia la velocità di accrescimento e varia la modalità di investimento delle risorse verso le radici o verso le riserve.\\
L'origine (gamica o agamica) fa variare la longevità (inferiore nel ceduo rispetto alla fustaia).\\
Le specie pioniere, avendo una velocità di crescita alta, hanno una bassa longevità.

\section{Struttura, successioni e dinamica forestale}
\subsection{Struttura e popolamento}
\textbf{Struttura:} distribuzione delle diverse parti dell'albero nello spazio forestale. La struttura può essere spaziale o temporale.\\
La struttura spaziale definisce l'organizzazione verticale ed orizzontale di un popolamento forestale.\\
La \textit{struttura verticale} è formata da una serie di strati di chiome di diversa altezza dal suolo, con diversa densità e diversa distanza reciproca; viene anche detta ``stratificazione". E' di fatto presente una gerarchia, dove domina la pianta che riesce ad assorbire maggiore radiazione.\\
Nei boschi misti la struttura verticale è generalmente pluristratificata, con gli strati inferiori costituiti da specie tolleranti l'ombra (sciafile). Nei boschi puri, la struttura verticale può essere rappresentata dalla coesistenza di più strati arborei (nel caso di specie tolleranti l'ombra) o da una monostratificazione (nel caso di specie eliofile).\\
La \textit{struttura orizzontale} (tessitura) è più complessa della verticale. Può essere di varie tipologie: casuale, per aggregati o per gruppi.\\
Questa distribuzione può dipendere da vari elementi: tipo di mescolanza specifica, microclima, macro/micro morfologia del terreno, modalità di propagazione della specie e dal tipo di disseminazione. La struttura orizzontale può costituire anche un adattamento a particolari condizioni ambientali.\\
La \textit{struttura temporale} indica come varia nel tempo il bosco, a seconda dei diversi periodi dell'anno (fenologia).
I processi di nascita, crescita, sviluppo e morte determinano il dinamismo delle popolazione e delle comunità.\\
\textbf{Popolamento:} gruppo di alberi, con una struttura ed una composizione uniforme, che accresce nelle stesse condizioni climatiche e di suolo.\\

\subsection{Biodiversità}
La biodiversità è di 3 livelli principali: genetica, specifica ed ecosistemica.\\
La diversità genetica è la meno considerata, però possiede la storia del popolamento passato.\\
La diversità specifica è rappresentata dal numero di specie, dalla loro equipartizione nel popolamento, dall'eterogeneità della specie,\dots\\
La diversità ecosistemica quantifica il livello di biodiversità del sito.\\
Le statistiche di diversità sono misurate per indici di ricchezza, ed indicano le componenti della biodiversità (necromassa, n° di specie,\dots).

\subsection{Dinamica forestale}
Il disturbo naturale è un qualunque evento discreto nel tempo che altera la struttura di un ecosistema e ne modifica la disponibilità di risorse.\\
La \textbf{successione} è il processo attraverso il quale le specie occupano un ambiente fisico e ne determinano le modificazioni.\\
Può essere di diverse tipologie:
\begin{itemize}
    \item primaria: ha inizio su un substrato che in precedenza era privo di copertura vegetale (es. dune del deserto);
    \item secondaria: ricostituzione della vegetazione dopo che la copertura vegetale è stata distrutta (es. processo di naturale forestazione dei terreni agricoli in abbandono);
    \item autogena: i cambiamenti del popolamento sono indotti da processi biologici che avvengono all'interno della comunità vegetale (es. competizione);
    \item allogena: i cambiamenti del popolamento sono indotti da processi esterni (es. disturbi abiotici naturali o antropici);
    \item a staffetta: successione nel tempo dei popolamenti, causa il cambio del clima. Le specie successive alla prima non sono presenti nel primo periodo;
    \item inizio comune: tutte le specie sono presenti sin dall'inizio, ma non tutte si trovano in condizioni favorevoli per esprimere il proprio potenziale. Quando le condizioni sono idonee, le piante iniziano ad occupare lo spazio disponibile. 
\end{itemize}
I processi di successione descrivono le relazioni tra le specie sulla probabilità di insediamento di altre specie:
\begin{itemize}
    \item facilitazione: una specie aiuta l'altra (es. leguminose e la fissazione dell'azoto);
    \item inibizione: una specie ostacola l'insediamento di altre (es. noce ed il rilascio di sostanze allelopatiche);
    \item tolleranza: le nicchie ecologiche delle specie non si sovrappongono, non creando alcun processo di facilitazione o inibizione.
\end{itemize}

La coorte è l'insieme di piante che si sono insediate in un determinato sito, in seguito ad un determinato evento (disturbo). E' possibile che gli individui della coorte abbiano età differenti (comunque con una differenza molto ridotta); infatti, è probabile che la germinazione possa essere avvenuta in periodi diversi.\\
A seconda del disturbo possono formarsi diverse strutture: fustaia coetanea, disetanea, coorti marginali\dots
\subsubsection{Stadi della dinamica forestale}
La dinamica forestale segue 4 step:
\begin{itemize}
    \item rinnovazione;
    \item competizione;
    \item stabilizzazione;
    \item senescenza.
\end{itemize}
Nello studio della dinamica forestale occorre tenere in considerazione la biomassa viva e la biomassa morta (necromassa).\\
In seguito ad un disturbo, la quantità di necromassa è alta. Con il tempo, avviene la decomposizione e tale valore si riduce.
\subsubsection{Ruolo del temperamento e della successione della specie}
A seconda della specie vegetale cambia il processo di successione:
\begin{itemize}
    \item sciafile: possono vivere con livelli di illuminazione relativamente bassi (sopportano l'ombreggiamento), sono tipiche degli stadi maturi;
    \item eliofile: devono vivere in piena luce e non possono vivere all'ombra di altre specie; sono tipiche dei primi stadi di successione (specie pioniere);
    \item pioniere: tendenzialmente eliofile, la produzione di semi è continua ed abbondante (con dispersione anemofila), i semi possono rimanere dormienti nella seed bank e germinano su un suolo preferibilmente minerale; hanno un accrescimento iniziale rapido, sono poco longeve e hanno una strategia R (massimizzazione della riproduzione);
    \item definitive: sono specie longeve (investendo le risorse sulle radici), producono pochi semi e di grandi dimensioni (annate di pasciona, con dispersione mediante gli animali), l'accrescimento è relativamente lento, la loro strategia è K.
\end{itemize}
\subsubsection{Foreste vetuste}
Le definizioni sono molte, con diversi principi di base: assenza di disturbi antropici, età minima; stadio di vetustà, passaggio dell'optimun di asportazione\dots
Sono particolari tipologie di boschi che hanno raggiunto in gran parte lo stato di vetustà. 
Le caratteristiche sono:
\begin{itemize}
    \item popolamento prettamente disetaneo, con grande diversificazione verticale ed orizzontale (determinata da eventi naturali);
    \item  presenza di mortalità per ogni classe di età;
    \item grande quantità di necromassa;
    \item alle nostre latitudini, il periodo di formazione è di 300 anni, mentre alle alte latitudini cala a 150 anni;
    \item le specie sono natie/indigene, con chiome profonde, ampie ed irregolari;
    \item hanno una superficie sufficiente per mantenere la popolazione, anche in caso di disturbi naturali;
\end{itemize}
Il ciclo forestale delle foreste vetuste è rappresentato dalla sovrapposizione di una fase di crescita, una fase ottimale ed una senescenza.\\
Gli schianti sono l'unica posizione per far germogliare e far crescere nuove piante (rinnovazione).

\subsection{Rinnovazione forestale}
La propagazione delle specie forestali può avvenire per via:
\begin{itemize}
    \item gamica (riproduzione): il patrimonio genetico è ereditato, ma non identico a quello dei genitori;
    \item agamica (moltiplicazione): il patrimonio genetico è identico a quello dei genitori.
\end{itemize}
La moltiplicazione attraverso le tecniche tradizionali è resa possibile dalla presenza di meristemi primari svernati e perennati (quindi non possibile nelle piante annuali).\\
La riproduzione avviene solo a maturità della pianta, che può essere condizionata della posizione, gerarchia,\dots Le piante dominanti hanno maggiore capacità di disseminazione del seme e di farsi impollinare.\\
Il fenomeno della riproduzione ha inizio in primavera, con eccezioni (alcune sono più precoci ed alcune più tardive).
\begin{table}[h]\centering 
\begin{tabular}{p{1.4cm}p{4cm}p{4cm}p{4cm}}
 & Naturale & Artificiale & Mista \\
 \midrule
Vantaggi  &  meno oneroso, si rinnovano specie adatte al sito, piante stabili & controllo della mescolanza e del periodo di rinnovazione   &  costi intermedi, applicabile in situazioni difficili, controllo parziale della mescolanza     \\
Svantaggi &le previsioni sono difficili, limitazioni esterne e disponibilità di seme  & costi elevati (soprattutto in siti difficili), logistica & svantaggi di entrambe le modalità      
\end{tabular}
\end{table}
\subsubsection{Impollinazione}
Avviene dall'incontro dei genitori, mediante diversi vettori.\\
Sono di diversi tipi:
\begin{itemize}
    \item anemofila (vento): nelle nostre foreste prevale questa tipologia;
    \item entomofila (insetti): è considerata come la forma di impollinazione più evoluta, prevalente nelle foreste equatoriali.
\end{itemize}
Ogni vettore ha le proprie caratteristiche e peculiarità; di conseguenza, le piante avranno caratteristiche particolari, proprio in base al vettore di impollinazione.
\subsubsection{Disseminazione}
Dispersione da parte dell'albero (e non) dei semi.\\
E' di diverse tipologie:
\begin{itemize}
    \item autocoria: autodisseminazione, effettuata dal frutto, senza bisogno di energie esterne (con la semplice caduta per gravità). L'albero tenderà a disseminare il seme più lontano possibile da sè;
    \item anemocoria: dispersione causata dal vento, i semi sono leggeri e di piccole dimensioni;
    \item idrocoria: dispersione dei semi per mezzo dell'acqua;
    \item zoocoria: dispersione causata dagli animali;
    \item antropocoria: dispersione procurata dall'uomo, mediante disseminazione intenzionale o diretta dei semi.
\end{itemize}
Il \textbf{successo della rinnovazione forestale} è dovuto a tre step: disponibilità e qualità di seme, insediamento/nascita e sopravvivenza/natalità.\\
Le condizioni di germinazione ideali sono: luce, temperatura ed umidità.\\
La rinnovazione naturale è caratterizzata da: complessità, irregolarità ed aleatorietà; i suoi tempi non sono certi.
\subsubsection{Rinnovazione artificiale}
Tramite semina diretta o mediante piantagione (trapianto).\\
La semina diretta offre:
\begin{itemize}
    \item la migliore garanzia per la stabilità a livello di singola pianta;
    \item è più compatibile con i processi biologici naturali;
    \item non avviene alcun shock da trapianto;
    \item i cosi sono contenuti ed inferiori rispetto alla piantumazione.
\end{itemize}
Purtroppo, la probabilità di successo è bassa ed imprevedibile. 
\subsubsection{Moltiplicazione}
Replicazione per via vegetativa di un individuo identico al genitore; in questo modo si mantiene inalterato il corredo genetico e le sequenze dei geni. Si basa sulla totipotenza delle cellule: il taglio ``libera" le gemme dagli ormoni prodotti dagli apici.\\
Le tecniche tradizionali si basano sull'impiego di gemme singole o multiple. Le gemme possono essere proventizie (polloni veri) o avventizie (falsi polloni); inoltre, ci sono i polloni radicali, che si generano in seguito ad un disturbo sulle radici.\\
Funziona bene per le latifoglie e meno per le conifere.\\
Le tecniche di moltiplicazione sono:
\begin{table}[h] \centering
\begin{tabular}{ccc}
 & senza distacco & con distacco \\
 \midrule
per gemma & ceduo & innesto per gemma \\
\midrule
per ramo & margotta, propaggine & talea, innesto per marza
\end{tabular}
\end{table}\\
La margotta consiste nel far radicare un ramo, finché è ancora radicato alla pianta madre. Di fatto si stimola la gemma ad emettere delle nuove radici.\\
La propaggine consiste nel piegare e far passare sotto il suolo un ramo giovane, in modo da farlo radicare.

\section{Selvicoltura}
Il principale strumento tecnico delle selvicoltura in bosco è l'eliminazione di alcune piante, così da favorire la crescita di altre piante. In altri casi, fa uso di impianto di alberi.\\
Il taglio di alcuni individui orienta i rapporti di competizione nel popolamento, in modo da favorire alcuni individui.\\
Alcune terminologie:
\begin{itemize}
    \item i sistemi selvicolturali sono l'insieme delle operazioni attuate per la coltivazione, l'utilizzazione e rinnovazione del bosco;
    \item il governo è il modo in cui il bosco di rinnova (ceduo e fustaia);
    \item i trattamenti sono azioni per regolare e rinnovare il bosco (per esempio i tagli di rinnovazione, intercalari,\dots);
    \item la conversione è il cambio di forma di governo;
    \item la trasformazione è il cambio di trattamento;
    \item il turno è il periodo che intercorre tra due utilizzazioni del soprassuolo maturo.
\end{itemize}
La durata del turno può essere scelta mediante quattro criteri:
\begin{itemize}
    \item fisiocratico: coincide con l'età in cui culmina l'incremento medio (massima produzione di biomassa per unità di superficie e di tempo);
    \item finanziario: basato su criteri finanziari (es. se i tassi di interesse sono alti conviene asportare materiale dal bosco);
    \item tecnico: garantisce e/o massimizza la produzione di determinati assortimenti legnosi;
    \item consuetudinari: si basa su tradizioni o usanze locali.
\end{itemize}
\subsection{Descrizione della stazione e del popolamento}
La stazione è un'area topograficamente definita, sulla quale dominano condizioni ecologicamente uniformi; inoltre, è caratterizzata da una stessa vegetazione naturale (potenziale), sulla quale è possibile applicare lo stesso trattamento selvicolturale.\\
Considerando la stazione, il clima ed il suolo è possibile compiere un'analisi stazionale, andando a considerare ogni parametro utile (area basimetrica, componente specifica, pluviometria\dots).\\
Il parametro ambientale che viene considerato maggiormente è il ``pluviofattore di Lang", che mette in relazione la precipitazione totale annua e la temperatura media annua di qualsiasi luogo.\\
E' possibile descrivere lo stadio evolutivo dei popolamenti coetaneiformi, secondo:
\begin{itemize}
    \item novelleto: insediamento di novellame, concorrenza tra vegetazione erbacea ed arbustiva, fino a 2 m di altezza;
    \item spessina: contatto tra le chiome, forte concorrenza intraspecifica (si differenzia la posizione sociale), avviene l'autopotatura dei rami più bassi, fino a 8-10 m;
    \item perticaia: forti incrementi diametrici e longitudinali, evidenti differenziazioni sociali, l'avviene il portamento ``forestale" dei dominanti, forte competizione e mortalità dei dominati, fino a 15-20 m e $<17,5$ cm di diametro;
    \item fustaia: riduzione dell'accrescimento longitudinale e diametrico, fase di assestamento (bassa competizione), riduzione della mortalità, il piano dominante è evidente, oltre i 20 cm di diametro.
\end{itemize}
Gli incrementi presi in considerazione per la descrizione possono essere ``correnti" (se riferiti all'ultima stagione vegetativa) oppure ``medi".

\section{Il governo a fustaia}
\subsection{La fustaia coetanea}
La fustaia coetanea ha un ciclo molto semplice: crescita, taglio, rigenerazione e riinizio del ciclo. Le cure svolte sono diverse, e servono per migliorare la produzione.\\
L'insediamento è condizionato dal taglio precedente (in questo caso è il taglio raso).\\
Durante il turno, vengono svolti degli interventi selvicolturali: i tagli intercalari, il taglio raso ed i tagli successivi.\\
In questo popolamento, la pianta ideale deve avere una ragionevole quantità di chioma: non eccessiva (perché sennò il gran numero di nodi nel tronco ne diminuirebbe il valore economico) e nemmeno eccessivamente ridotta (poiché ne risentirebbe la stabilità). La quantità di chioma viene regolata mediante i trattamenti selvicolturali. 
\subsubsection{Taglio raso}
Consiste nell'asportazione di tutte le piante presenti su una superficie (chiamata ``tagliata"). Raggiunta la maturità del bosco, si interviene tagliando tutto il soprassuolo con un unico intervento (il terreno rimane completamente scoperto).\\
In ecologia, il taglio raso è tale se ``riesce ad alterare completamente il microclima di un certo luogo/bosco (temperatura, proprietà mitigatrici,\dots)".\\
In questo trattamento non occorre individuare le piante da tagliare, bensì le superfici interessate; tali popolamenti sono marcatamente coetanei. \\
La rinnovazione è posticipata e può essere artificiale (rimboschimento) o naturale (mista). La ``pre-rinnovazione" è la rinnovazione già presente in bosco, ancora prima del trattamento, e non ha alcun interesse (a meno di particolari specie).\\
In Italia, il taglio raso è vietato se interessa superfici di fustaia superiori a 5000 $m^2$. Vengono autorizzati nei casi di particolari condizioni, come per esempio per la difesa fitosanitaria o per il ripristino post-incendio: in questi casi devono essere garantite le azioni di rinnovazione naturale o artificiale del bosco.\\
La superficie di interesse del taglio raso può essere:
\begin{itemize}
    \item grande superficie ($>1$ ha);
    \item piccola superficie ($<1$ ha);
    \item taglio a buche (1000-1500 $m^2$): apertura di piccole buche, dipendenti dall'altezza degli alberi (per motivi di disseminazione), in questo metodo viene copiata l'azione dei disturbi naturali;
    \item taglio raso a strisce: taglio che porta alla creazione di radure lineari, con lunghezza e larghezza variabili;
    \item taglio a fessure: molto più stretto di quello raso a strisce;
    \item taglio marginale: la striscia tagliata si trova al bordo del bosco (margine);
    \item taglio ad orlo: come per il marginale, il taglio avviene su un margine già provvisto di rinnovazione;
    \item taglio raso commerciale: non presente in Italia;
    \item taglio raso con riserve: vengono rilasciati alcuni individui, per la disseminazione e per la difesa della biodiversità;
    \item taglio raso a quinte: come il precedente, ma riguarda gruppi di alberi e non singoli individui.
\end{itemize}
Il taglio raso simula i disturbi naturali (fuoco, vento,\dots). Al contrario dei disturbi, questo metodo prevede la rimozione della biomassa a terra.\\
E' ideale nel caso si volesse coltivare specie pioniere o eliofile.\\
La meccanizzazione è spinta, e per questo motivo i costi sono ridotti. La gestione risulta semplice.\\
L'impatto ecologico è alto (erosione, paesaggio, fauna), c'è scarsa elasticità di gestione, e gli investimenti/ricavi sono lontani nel tempo.
\subsubsection{Tagli intercalari}
Gli obiettivi di questo trattamento sono:
\begin{itemize}
    \item regolare la densità degli individui, per concentrare la crescita solamente nelle piante di interesse selvicolturale (in modo da ridurre le morti da competizione);
    \item produrre assortimenti legnosi differenti rispetto a quelli che si avrebbero a fine turno;
    \item aumentare la stabilità degli alberi, in modo da garantirgli l'arrivo a fine ciclo;
    \item cambiare le condizioni del bosco, in modo da far arrivare più risorse al suolo.
\end{itemize}
Questo trattamento non crea rinnovazione, però ne facilità il futuro insediamento (migliorando la lettiera, scaldando il terreno,\dots).\\
A seconda del periodo, il taglio intercalare prende diversi nomi:
\begin{itemize}
    \item novelleto e spessina: sfolli e ripuliture;
    \item perticaia e fustaia: diradamenti.
\end{itemize}
Gli sfolli e le ripuliture sono considerati a selezione negativa, poiché si vanno a rimuovere elementi che non hanno interesse economico/selvicolturale. I diradamenti sono considerati a selezione negativa o positiva, a seconda degli individui che vengono rimossi.\\
I diradamenti possono essere identificati da: tipo, grado, età di inizio e frequenza. Il tipo è un parametro qualitativo, mentre il grado è un parametro quantitativo; l'età di inizio e la frequenza identificano l'inquadramento temporale del diradamento.\\
I tipi di diradamento posso essere:
\begin{itemize}
    \item dal basso (tedesco): intervengono nelle piante dominate del bosco, non è molto efficace (le piante di interesse hanno già perso la competizione);
    \item dall'alto: intervengono nel piano dominante e codominante, comporta scelte importanti ed è molto efficace (poiché permette l'entrata in bosco di molta luce);
    \item misto: poiché interessano entrambi i piani;
    \item liberi: scelte sull'avvenire di singoli alberi;
    \item selettivi: dopo aver individuato l'albero da portare a fine turno, lo si andrà ad ``educare", asportando individui in quella relativa cella.
    \item meccanico (o schematico): segue un ordine geometrico e geografico di taglio, secondo scelte preventive, i tagli possono essere lineari o a zone.
\end{itemize}
Il grado di diradamento può essere: 
\begin{itemize}
    \item debole: non viene alterata la copertura all'interno del popolamento;
    \item moderato;
    \item forte;
    \item fortissimo: viene interrotta la copertura all'interno del popolamento.
\end{itemize}
Il diradamento dal basso può essere debole, moderato, forte e superdiradamento. Il diradamento dall'alto può essere debole o forte (poiché agisce sul piano dominante).\\
E' importante che il grado di diradamento non sia costante durante tutta la vita del popolamento. Generalmente il grado è più pesante quando le piante sono più giovani, ovvero più reattive.\\
Un altro parametro che identifica il diradamento è l'epoca di inizio: è precoce si viene svolto prima del culmine della crescita in altezza, mentre è tardivo se viene svolto dopo la culminazione in altezza.\\
La frequenza del diradamento indica ogni quanto tempo vengono applicati i trattamenti di diradamento: è detto frequente se vengono svolti ad intervalli inferiori al decimo del turno. Frequenza e grado si condizionano a vicenda: all'aumentare della frequenza andrà a calare il grado.

\subsubsection{Tagli successivi}
Liberazione del soprassuolo con più tagli successivi, mantenendo quindi una certa copertura. L'obiettivo di questo trattamento, oltre al fine economico, è quello di ottenere nuovamente un popolamento coetaneiforme (ovvero rinnovazione naturale ottenuta nello stesso momento).\\
Vengono mantenuti alberi per:
\begin{itemize}
    \item non stravolgere il clima del bosco;
    \item disseminazione.
\end{itemize}
Ci sono diversi tagli successivi: 
\begin{enumerate}
    \item taglio di sementazione: avviene a fine turno, quando è stato raggiunto lo scopo selvicolturale. Con questo taglio avviene l'inizio della rinnovazione. Viene asportato un volume proporzionale alla quantità e tipologia di piante (30\% se sciafile e 70\% se eliofile). Le piante rilasciate produrranno seme, ovvero condizioneranno la rinnovazione;
    \item tagli secondari: effettuati tra il taglio di sementazione e quello di sgombero, accompagnano l'insediamento e la crescita graduale della rinnovazione, non sono obbligatori;
    \item taglio di sgombero: è quello conclusivo con il quale si asporta tutto il soprassuolo residuo.
\end{enumerate}
Il periodo tra il taglio di sementazione e quello di sgombero è chiamato ``periodo di rinnovazione". Se tale periodo è eccessivo, risulta difficoltosa la rimozione degli alberi rimasti, essendo già presente rinnovazione; inoltre, è più probabile la formazione di un popolamento disetaneiforme. Per questi motivi, minore è tale periodo e meglio è.\\
Questi tagli successivi possono essere considerati come dei diradamenti.\\
Inoltre, possono interessare un'intera area oppure piccole porzioni di terreno (in tal caso saranno tagli a gruppi, strisce, orlo ed a gruppi e strisce).\\
Nel caso di tagli su singole porzioni, occorre ragionare sulla logistica da seguire per l'asportazione del materiale.\\
Questo trattamento imita i disturbi naturali di modesta entità e su piccole superfici. Al contrario del taglio raso, è possibile scegliere quali alberi rimuovere e quali lasciare per la rinnovazione.\\
I vantaggi sono:
\begin{itemize}
    \item copertura continua del suolo;
    \item rinnovazione naturale;
    \item selezione del portamento;
    \item possibilità di meccanizzare le operazioni;
    \item produzione di differenti assortimenti legnosi.
\end{itemize}
Gli svantaggi sono:
\begin{itemize}
    \item difficoltà nella gestione e scarsità di elasticità;
    \item problemi con la rinnovazione insediata durante i tagli secondari ed il taglio di sgombero;
    \item investimenti e ricavi lontani nel tempo.
    \end{itemize}

\subsection{La fustaia disetanea} 
Per definizione: ``su superficie ridotta è possibile trovare alberi con età e dimensioni differenti". Quindi, è possibile trovare rinnovazione, alberi intermedi e piante mature.\\
Essendo un popolamento diversificato, è per definizione molto resiliente e resistente (maggiore stabilità strutturale).\\
Le caratteristiche principali sono:
\begin{itemize}
    \item presenza di molte dimensioni ed età differenti;
    \item piena occupazione del piano verticale, con chiome irregolari;
    \item non dovrebbero esserci distribuzioni spaziali regolari;
    \item la rinnovazione è continua e naturale, il novellame presente è sufficiente a compensare le perdite;
    \item la produttività resta costante nel tempo, con piccole variazioni dovute alle variazioni di provvigione;
    \item non ci sono differenziazioni microecologiche nel tempo, ma ci sono differenti microclimi presenti;
    \item la fertilità del sito viene mantenuta costante;
    \item c'è maggiore scelta del diametro degli alberi (data la disomogeneità presente).
\end{itemize}
Al contrario delle foreste vetuste, non sono presenti piante di grandissima dimensione e molta necromassa.\\
Dove c'è già rinnovazione, la crescita in seguito ad un disturbo è già garantita.\\
Nelle foreste gestite dall'uomo, molto probabilmente la curva di distribuzione dei diametri è perfetta. Inoltre, gli uomini gestiscono questi popolamenti imitando le foreste poco disturbate (di piccola entità).\\
Al fine di ottenere un guadagno economico, occorre attuare una selvicoltura intensiva, oltre ad avere una certa coltura e professionalità.
\subsubsection{Taglio saltuario}
E' il trattamento che viene applicato alla fustaia disetanea. Oltre al prelievo (per fini economici), si ``educa" il popolamento, ovvero si cerca di mantenere costante nel tempo la disetaneità del bosco.\\
Permette l'asportazione di legname, senza dover aspettare la fine del turno. Infatti, non esiste il turno, bensì il ``periodo di curazione", ovvero il tempo che intercorre tra un taglio di curazione ed un altro.\\
E' concorde con le specie più sciafile, poiché' la loro crescita è (rispetto alle eliofile) superiore sotto copertura. Il taglio saltuario a gruppi invece favorisce le piante eliofile, poiché viene interrotta maggiormente la copertura delle chiome.\\
Generalmente, si dovrebbe asportare per ogni taglio un 15-25\% di massa del bosco. Se la percentuale di asporto è troppo esigua, viene impedita l'affermazione della rinnovazione (coetanizzazione dall'alto); se la martellata risulta troppo pesante, verrà a crearsi una grande rinnovazione in modo contemporaneo (coetanizzazione dal basso). In entrambi i casi, ci sarà la tendenza alla creazione di una fustaia monoplana.\\
Un errore di coetanizzazione dall'alto richiede molti anni per essere corretto; risulta più semplice correggere un errore di coetanizzazione dal basso.\\
Al fine di evitare i problemi di coetanizzazione, questo trattamento viene svolto assieme al "metodo di controllo", ovvero il cavallettamento nel tempo del popolamento, in modo da conoscere l'andamento del bosco.\\
Teoricamente sarebbe possibile entrare ogni anno in bosco per rimuovere gli assortimenti legnosi; questa tecnica risulterebbe poco vantaggiosa economicamente. Per tale motivo, si impone come periodo di curazione quello che impiega mediamente una pianta della penultima classe diametrica ad arrivare alla successiva (ovvero quella di interesse selvicolturale).\\
I vantaggi maggiormente apprezzati del taglio saltuario sono ecologico-ambientali, poiché c'è sempre copertura al suolo (e quindi protezione da valanghe, frane,\dots).\\
Gli svantaggi del taglio saltuario sono:
\begin{itemize}
    \item assenza di necromassa (e quindi assenza di dendromicrohabitat);
    \item assenza di piante di grandi dimensioni;
    \item la correzione degli errori richiede molto tempo;
    \item il degrado del popolamento si crea dalla continua rimozione delle migliori piante (lasciando le peggiori a svolgere il ruolo di semenzale);
    \item non avviene il cambiamento delle specie (rimangono sempre le stesse).
\end{itemize}
Il taglio saltuario imita i disturbi naturali di piccola grandezza, solitamente riferiti a singoli o pochi alberi.
\section{Il governo a ceduo}
La rinnovazione di questo tipo di bosco avviene per via agamica, mediante la capacità pollonifera della pianta (la rinnovazione è naturale); in seguito al taglio dell'assortimento legnoso, la pianta ricaccia una serie di polloni.\\
Viene utilizzata la frazione aerea della pianta. Rispetto alla fustaia, la chioma della pianta a ceduo tende a crescere verso l'esterno, generando una sciabolatura (e rendendo meno nobile l'assortimento).\\
I prodotti legnosi ricavati da questo governo sono principalmente utilizzati per la combustione o per la piccola paleria.\\
Le gemme proventizie generano i polloni veri, ovvero sono in grado di ricostruire la chioma in seguito ad eventi traumatici. Le gemme avventizie si formano in corrispondenza di un trauma, direttamente sul callo cicatriziale, e creeranno dei falsi polloni.\\
Dal punto di vista gestionale, il pollone più basso avrà più tendenza di creare nuove radici, e quindi una nuova pianta.\\
Le latifoglie hanno la capacità di ricacciare, che può essere un problema nel caso che si volesse eliminare tali piante (soprattutto se invasive). Anche le conifere, presenti in certi ambienti, sono in grado di ricacciare.\\
I polloni possono essere:
\begin{itemize}
    \item caulinari: nati sul fusto o nella ceppaia;
    \item radicali: originato da una radice ed emerge dal terreno (es. robinia);
    \item affrancati.
\end{itemize}
Le differenze tra una pianta nata da seme o da pollone sono:
\begin{itemize}
    \item accrescimento: è più sostenuto nelle piante nate da pollone, poiché l'apparato radicale è già presente. Alla lunga, il seme può superare in velocità il pollone;
    \item foglie: sono più piccole nelle piante nate da seme, rendendo inferiore la produzione di fotosintesi;
    \item longevità: essendo inversamente proporzionale alla velocità, è superiore per le piante nate da seme;
    \item maturità sessuale: tendenzialmente, avviene prima nelle piante nate da pollone, creando però semi di quantità e qualità inferiori;
    \item corteccia: è più liscia nelle piante nate da pollone;
    \item legno: nelle piante nate da pollone la densità è inferiore, poiché la crescita è più veloce.
\end{itemize}
Dal punto di vista gestionale, il parametro più importante è la velocità di crescita, in modo da ridurre i tempi di trattamento ed asportazione.\\
Ci sono diverse tipologie di ceduazione:
\begin{itemize}
    \item a ceppaia: il taglio avviene in prossimità del suolo; può essere ``fuori terra" (5-20 cm dal suolo), succisione (rasente al suolo) o tramarratura (tra due terre);
    \item a capitozza: il taglio avviene a 1-3 m di altezza;
    \item a sgamollo: il taglio interessa solamente i rami laterali dell'albero, lasciando intatto l'apice vegetativo.
\end{itemize}
Questi ultimi due tipi di ceduazione non sono più presenti in bosco, ma solamente in alcuni contesti (es. ville, bordi strada,\dots).
Nel ceduo a capitozza, il fusto può essere considerato come la ceppaia, ed i rami possono essere considerati come dei polloni.\\
Il castagno viene considerato come se avesse una capacità pollonifera quasi infinita (al contrario del faggio).\\
Le utilizzazioni del ceduo devono essere suddivise in base ai livelli quantitativi dell'intervento stesso:
\begin{itemize}
    \item max 2000 mq di piccoli tagli boschivi (per autoconsumo): nessun vincolo e nessuna comunicazione;
    \item max 2,5 ha: viene richiesta la dichiarazione di taglio, svolta da parte del proprietario;
    \item oltre i 2,5 ha: viene richiesta la dichiarazione di taglio ed il progetto di taglio, svolti entrambi da un dottore forestale.
\end{itemize}
Per piccoli tagli fitosanitari non è necessario effettuare avvisi o richieste.\\
Al fine di effettuare una buona ceduazione, occorre seguire quattro indicazioni importanti:
\begin{enumerate}
    \item tagliare bene: la corteccia non dev'essere slabbrata, con un taglio convesso (in modo da non accumulare acqua);
    \item tagliare basso: in modo da non alzare la ceppaia, e di conseguenza perdere la capacità pollonifera;
    \item tagliare presto, ma non troppo: la capacità pollonifera della pianta non è infinita, ma occorre che ci siano abbastanza gemme pollonifere;
    \item riposo vegetativo: da svolgere nel periodo silvano (ovvero nel periodo senza attività vegetativa), essendo le sostanze nutritive presenti nella ceppaia. Inoltre, il taglio in tale periodo permette la lignificazione in primavera-estate, permettendo la sopravvivenza nella stagione successiva.
\end{enumerate}
Il governo a ceduo possiede:
\begin{itemize}
    \item vantaggi: cicli brevi, copertura del terreno continua, rinnovazione certa e facilità di gestione;
    \item svantaggi: semplificazione dell'ecosistema, costi di gestione superiore, impatto sul paesaggio e mancanze di nicchie ecologiche.
\end{itemize}
Il ceduo può essere trattato in tre modi: semplice, matricinato ed a sterzo.

\subsection{Ceduo semplice}
Avviene il taglio a raso del soprassuolo maturo, rimuovendo tutti i polloni della particella.\\
La struttura che ne deriva sarà coetaneiforme, proprio perché il taglio avviene nello stesso momento.\\
Non è possibile applicarlo a tutte le specie, ma solo a quelle con un'alta capacità pollonifera.\\
Il turno è molto breve (5-6 anni).
\subsection{Ceduo matricinato}
Oltre alla presenza della ceppaia (che ricaccia i polloni e che viene trattata a raso), è presente una matricina, ovvero una pianta idealmente nata da seme. Le matricine presenti vengono trattate mediante il taglio saltuario.\\
La presenza di alcuni polloni che hanno un'età superiore al turno permette il rimpiazzo o la sostituzione delle ceppaie morte, quindi mediante rinnovazione naturale da seme.\\
E' importante che le matricine non siano troppo vecchie, poiché avverrebbe l'ombreggiamento delle ceppaie.\\
Nel momento del taglio vengono rilasciati alcuni ``allievi", che diventeranno matricine; quest'ultime hanno un'età massima di 2-3 volte quella dei polloni. \\
Il numero di matricine rilasciate in bosco è normato dalla legge.\\
La presenza di matricine in bosco permette di aumentare il grado di biodiversità, specialmente se sono di specie diverse, e permette la creazione di assortimenti legnosi di diametro ed altezza superiori rispetto a quelli del ceduo semplice.\\
Pur essendoci degli individui che vengono mantenuti in bosco, la copertura e protezione del suolo risulta insufficiente.
\subsection{Ceduo a sterzo}
E' una tipologia di bosco disetaneo, dove i polloni presenti sulla ceppaia hanno età diversa.\\
Viene trattato mediante taglio saltuario (come per la fustaia disetanea). La distribuzione diametrale dei polloni è simile a quella che si trova nella fustaia disetanea.\\
Avendo un certo tempo di turno, il trattamento viene diviso per tre periodi, dove per ognuno viene asportato un terzo del numero dei polloni. I polloni rimossi sono quelli più vecchi, lasciando i più giovani in bosco a crescere; in questo modo viene garantita una massa legnosa per ogni periodo.\\
Questo trattamento produce meno assortimenti rispetto al ceduo semplice, e di diversa grandezza.\\
In questo trattamento c'è sempre la copertura forestale, e perciò potrebbe essere usata come metodo di protezione. Data la sua continua disponibilità di assortimenti, è apprezzata per gli usi civici.
\section{La selvicoltura d'albero}
Invece di considerare una particella di bosco, vengono solamente considerati alcuni alberi ad ettaro (che produrranno assortimenti legnosi di qualità).\\
L'obiettivo principale è quello di ottenere un albero con un bel toppo; in questo caso, il portamento forestale non viene più considerato. Di conseguenza, l'autopotatura della chioma non è più necessaria, poiché si vuole ottenere una chioma grande, in modo che influenzi ed acceleri la crescita del fusto.\\
Le fasi della selvicoltura d'albero sono:
\begin{enumerate}
    \item insediamento della rinnovazione (naturale), tendenzialmente su boschi misti;
    \item qualificazione;
    \item dimensionamento;
    \item maturazione.
\end{enumerate}
Dopo aver scelto la pianta d'interesse forestale, la si libera dalla competizione successivamente alla culminazione della sua altezza. Questa riduzione della competizione fa aumentare l'accrescimento radiale. Quando la crescita relativa diminuisce, si può decidere se tagliare la pianta o lasciarla in bosco ulteriormente (sarebbe meglio minimizzare la frazione di alburno).\\
Il popolamento non può essere definito disetaneo, perché non segue tale andamento.\\
Questo trattamento garantisce la biodiversità, perché permette la presenza in bosco di diverse grandezze di piante.\\
\section{La selvicoltura di protezione}
E' una tipologia di trattamento che viene applicata nelle foreste di protezione, in modo da difendere abitati/strade dagli eventi naturali. In pratica, vengono applicate le ``cure minime", mirate per non ridurre la funzione protettiva.\\
Tali cure hanno le seguenti caratteristiche:
\begin{itemize}
    \item adeguate all'obiettivo di protezione;
    \item svolte nel luogo appropriato;
    \item applicate nel momento giusto;
    \item in sintonia con l'ambiente naturale;
    \item efficaci.
\end{itemize}
L'applicazione di questo trattamento risulta più economica rispetto alla creazione di infrastrutture di protezione permanenti.\\
\section{Il ceduo composto ed il governo misto}
Nel ceduo composto una parte del popolamento è a ceduo ed una parte è governata a fustaia. Il governo misto è simile al ceduo composto, ma va a cambiare la distribuzione spaziale degli alberi (maggiormente irregolare).\\
Nel ceduo composto la fustaia viene gestita come se fosse una foresta disetanea, mentre la parte a ceduo viene gestita a ceduo semplice o a sterzo. Gli interventi selvicolturali vengono svolti contemporaneamente.\\
Le matricine rilasciate hanno 5-6 volte l'età del turno del ceduo.\\
Il occasione del taglio, avviene il taglio a raso del ceduo, l'asportazione degli alberi ad alto fusto maturi ed il rilascio di un certo numero di matricine.\\
Il ceduo composto comporta:
\begin{itemize}
    \item copertura al suolo continua (anche dopo il taglio);
    \item elevato numero di specie;
    \item possibilità di svolgere il pascolo in bosco;
    \item continua produzione di legno (dato il basso ciclo del ceduo semplice).
\end{itemize}
D'altro canto, l'utilizzazione del ceduo composto è rigida; per questo motivo è stato introdotto il governo misto.
\section{Le conversioni e le trasformazioni}
La conversione è il cambio di governo (ovvero il metodo di rinnovazione). La trasformazione è il cambio del sistema selvicolturale (per esempio da ceduo semplice a ceduo matricinato).\\
Le conversioni non sono sempre permesse. Per legge è vietata la conversione da fustaia a ceduo (con eccezione per i motivi fitosanitari, di ricerca, tutela idrogeologica, lotta agli incendi,\dots).\\
In tutti i casi, la conversione da ceduo a fustaia necessita di un periodo di invecchiamento del bosco.\\
La conversione all'alto fusto dei boschi cedui può essere artificiale (diretta) o naturale (invecchiamento, matricinatura intensiva e matricinatura progressiva).\\
La conversione diretta richiede l'uccisione del ceduo e la semina/piantagione del nuovo soprassuolo.\\
La conversione per via naturale viene svolta in modo indiretto.\\
L'invecchiamento richiede (come dice il nome) l'invecchiamento del ceduo, in modo che perda la capacità pollonifera e che raggiunga la maturità sessuale.\\
La matricinatura intensiva richiede la ceduazione del vecchio soprassuolo, in modo da creare una ``fustaia transitoria"; a questa viene effettuato un taglio di sementazione e di sgombero, che porterà al termine la conversione. Questo metodo richiede tempi lunghi.\\
La matricinatura progressiva comporta la ceduazione continua degli individui, portando il bosco verso un ``ceduo composto transitorio"; seguirà una serie di tagli di sementazione e di sgombero. 
\end{document}